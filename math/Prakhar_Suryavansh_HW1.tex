\documentclass[12pt]{article}
\usepackage[margin=1in]{geometry}
\usepackage{titlesec}
\usepackage{setspace}
\usepackage{amsmath}
\usepackage{amsfonts}
\usepackage{graphicx}

\title{CSCE 633 - Homework 1}
\author{Prakhar Suryavansh}
\date{}

\begin{document}

\maketitle

\section*{\underline{Problem 1: Gradient Calculation}}

\subsection*{Part 1:}
Given the function:
\[
  f(x, y) = x^2 + \ln(y) + xy + y^3
\]
We need to find the gradient of this function at the point \((x, y) = (10, -10)\).

The gradient vector is given by:
\[
  \nabla f(x, y) = \left( \frac{\partial f}{\partial x}, \frac{\partial f}{\partial y} \right)
\]

1. Compute \(\frac{\partial f}{\partial x}\):
\[
  \frac{\partial f}{\partial x} = 2x + y
\]
Substitute \(x = 10\) and \(y = -10\):
\[
  \frac{\partial f}{\partial x} = 2(10) + (-10) = 20 - 10 = 10
\]

2. Compute \(\frac{\partial f}{\partial y}\):
\[
  \frac{\partial f}{\partial y} = \frac{1}{y} + x + 3y^2
\]
Substitute \(x = 10\) and \(y = -10\):
\[
  \frac{\partial f}{\partial y} = \frac{1}{-10} + 10 + 3(-10)^2 = -0.1 + 10 + 300 = 309.9
\]

Thus, the gradient at the point \((10, -10)\) is:
\[
  \nabla f(10, -10) = (10, 309.9)
\]

\subsection*{Part 2:}
Given the function:
\[
  f(x, y, z) = \tanh(x^3y^3) + \sin(z^2)
\]
We need to find the gradient of this function at the point \((x, y, z) = (-1, 0, \frac{\pi}{2})\).

The gradient vector is given by:
\[
  \nabla f(x, y, z) = \left( \frac{\partial f}{\partial x}, \frac{\partial f}{\partial y}, \frac{\partial f}{\partial z} \right)
\]

1. Compute \(\frac{\partial f}{\partial x}\):
\[
  \frac{\partial f}{\partial x} = 3x^2y^3 \text{sech}^2(x^3y^3)
\]
At \((x, y) = (-1, 0)\), this simplifies to:
\[
  \frac{\partial f}{\partial x} = 0
\]

2. Compute \(\frac{\partial f}{\partial y}\):
\[
  \frac{\partial f}{\partial y} = 3x^3y^2 \text{sech}^2(x^3y^3)
\]
At \((x, y) = (-1, 0)\), this simplifies to:
\[
  \frac{\partial f}{\partial y} = 0
\]

3. Compute \(\frac{\partial f}{\partial z}\):
\[
  \frac{\partial f}{\partial z} = 2z \cos(z^2)
\]
Substitute \(z = \frac{\pi}{2}\):
\[
  \frac{\partial f}{\partial z} = 2\left(\frac{\pi}{2}\right) \cos\left(\left(\frac{\pi}{2}\right)^2\right) = \pi \cdot \cos\left(\frac{\pi^2}{4}\right)
\]
Thus, the gradient at the point \((-1, 0, \frac{\pi}{2})\) is:
\[
  \nabla f(-1, 0, \frac{\pi}{2}) = (0, 0, \pi \cos(\frac{\pi^2}{4}))
\]

\section*{\underline{Problem 2: Matrix Multiplication}}

\subsection*{Part 1:}
Given the matrices:
\[
  \mathbf{A} = \begin{bmatrix} 10 \\ -5 \\ 2 \\ 8 \end{bmatrix}, \quad
  \mathbf{B} = \begin{bmatrix} 0 & 3 & 0 & 1 \end{bmatrix}
\]

Perform the matrix multiplication \(\mathbf{A} \times \mathbf{B}\):

\[
  \mathbf{A} \times \mathbf{B} = \begin{bmatrix}
    10 \cdot 0 & 10 \cdot 3 & 10 \cdot 0 & 10 \cdot 1 \\
    -5 \cdot 0 & -5 \cdot 3 & -5 \cdot 0 & -5 \cdot 1 \\
    2 \cdot 0  & 2 \cdot 3  & 2 \cdot 0  & 2 \cdot 1  \\
    8 \cdot 0  & 8 \cdot 3  & 8 \cdot 0  & 8 \cdot 1  \\
  \end{bmatrix}
  =
  \begin{bmatrix}
    0 & 30  & 0 & 10 \\
    0 & -15 & 0 & -5 \\
    0 & 6   & 0 & 2  \\
    0 & 24  & 0 & 8  \\
  \end{bmatrix}
\]

\subsection*{Part 2:}
Given the matrices:
\[
  \mathbf{C} = \begin{bmatrix}
    1  & -1 & 6 & 7 \\
    9  & 0  & 8 & 1 \\
    -8 & 1  & 2 & 3 \\
    10 & 4  & 0 & 1
  \end{bmatrix}, \quad
  \mathbf{D} = \begin{bmatrix}
    6  & 2  & 0 \\
    0  & -1 & 1 \\
    -3 & 0  & 4 \\
    3  & 4  & 7
  \end{bmatrix}
\]

Perform the matrix multiplication \(\mathbf{C} \times \mathbf{D}\):

Step-by-step calculation:
\[
  \mathbf{C} \times \mathbf{D} = \begin{bmatrix}
    1  & -1 & 6 & 7 \\
    9  & 0  & 8 & 1 \\
    -8 & 1  & 2 & 3 \\
    10 & 4  & 0 & 1
  \end{bmatrix}
  \cdot
  \begin{bmatrix}
    6  & 2  & 0 \\
    0  & -1 & 1 \\
    -3 & 0  & 4 \\
    3  & 4  & 7
  \end{bmatrix}
\]

First row of \(\mathbf{C} \times \mathbf{D}\):
\[
  1 \cdot 6 + (-1) \cdot 0 + 6 \cdot (-3) + 7 \cdot 3 = 6 + 0 - 18 + 21 = 9
\]
\[
  1 \cdot 2 + (-1) \cdot (-1) + 6 \cdot 0 + 7 \cdot 4 = 2 + 1 + 0 + 28 = 31
\]
\[
  1 \cdot 0 + (-1) \cdot 1 + 6 \cdot 4 + 7 \cdot 7 = 0 - 1 + 24 + 49 = 72
\]
First row: \([9, 31, 72]\)

Second row of \(\mathbf{C} \times \mathbf{D}\):
\[
  9 \cdot 6 + 0 \cdot 0 + 8 \cdot (-3) + 1 \cdot 3 = 54 + 0 - 24 + 3 = 33
\]
\[
  9 \cdot 2 + 0 \cdot (-1) + 8 \cdot 0 + 1 \cdot 4 = 18 + 0 + 0 + 4 = 22
\]
\[
  9 \cdot 0 + 0 \cdot 1 + 8 \cdot 4 + 1 \cdot 7 = 0 + 0 + 32 + 7 = 39
\]
Second row: \([33, 22, 39]\)

Third row of \(\mathbf{C} \times \mathbf{D}\):
\[
  -8 \cdot 6 + 1 \cdot 0 + 2 \cdot (-3) + 3 \cdot 3 = -48 + 0 - 6 + 9 = -45
\]
\[
  -8 \cdot 2 + 1 \cdot (-1) + 2 \cdot 0 + 3 \cdot 4 = -16 - 1 + 0 + 12 = -5
\]
\[
  -8 \cdot 0 + 1 \cdot 1 + 2 \cdot 4 + 3 \cdot 7 = 0 + 1 + 8 + 21 = 30
\]
Third row: \([-45, -5, 30]\)

Fourth row of \(\mathbf{C} \times \mathbf{D}\):
\[
  10 \cdot 6 + 4 \cdot 0 + 0 \cdot (-3) + 1 \cdot 3 = 60 + 0 + 0 + 3 = 63
\]
\[
  10 \cdot 2 + 4 \cdot (-1) + 0 \cdot 0 + 1 \cdot 4 = 20 - 4 + 0 + 4 = 20
\]
\[
  10 \cdot 0 + 4 \cdot 1 + 0 \cdot 4 + 1 \cdot 7 = 0 + 4 + 0 + 7 = 11
\]
Fourth row: \([63, 20, 11]\)

Thus, the result of \(\mathbf{C} \times \mathbf{D}\) is:
\[
  \mathbf{C} \times \mathbf{D} = \begin{bmatrix}
    9   & 31 & 72 \\
    33  & 22 & 39 \\
    -45 & -5 & 30 \\
    63  & 20 & 11
  \end{bmatrix}
\]

\end{document}